% LaTeX introduction practical 2/2

% Dr. Noël Juvigny-Khenafou
% University of Koblenz-Landau, Winter semester 2021

% In this practical we will create a rough template for an academic CV to go around some of the features of LaTeX.
% You can then later use this template and improve it for your own CV.
% Enjoy !

\documentclass[11pt, a4paper]{moderncv} % we define the type of document
\moderncvtheme[grey]{classic} % we set the theme. More colours and theme are available and I refer you to the documentation of the package for that.
\usepackage[english]{babel}

% package to manage bibliography
\usepackage[
backend=biber,
bibstyle=numeric,
citestyle=authoryear,
sorting=ydnt,
url=false,
isbn=false,
minbibnames=10,
maxbibnames=10
]{biblatex}

\addbibresource{cv_biblio.bib} % here is where we tell biblatex the name of the library

%To highlight one author:
% Run biber once, find the hash you want in the generated .bbl file 
% and paste here:

\newcommand*{\boldnames}{}
\newcommand*{\mkboldifhashinlist}[1]{%
  \xifinlist{\thefield{hash}}{\boldnames}
    {\mkbibbold{#1}}
    {#1}}

\DeclareNameWrapperFormat{boldifhashinlist}{%
  \renewcommand*{\mkbibcompletename}{\mkboldifhashinlist}%
  #1}

\DeclareNameWrapperAlias{sortname}{default}
\DeclareNameWrapperAlias{default}{boldifhashinlist}


\newcommand*{\detokenizelistadd}[2]{%
  \listeadd#1{\detokenize{#2}}}

\renewcommand*{\boldnames}{}
\forcsvlist{\detokenizelistadd\boldnames}
  {{ba76807aca2e7c8f36a0de7bf74e4fc7},{87c792a84d46355ede51faa39ef4432a},{1c3fff0738e5fb7fe82a955af6e85ee9}}   

% define the encoding
\usepackage[utf8]{inputenc} % enables to enter accent for Mac and Linux users
% for windows users use: \usepackage[latin1]{inputenc}

% adjust the page margins
\usepackage[scale=0.8]{geometry}
\recomputelengths

% enter personal data
\firstname{John}
\familyname{Snow}
\title{Curriculum vitae}
\address{123 The Wall }{XXXXX Westeros}
\mobile{+XXX XXX XXX}
\phone{+XXX XXX XXX}
\fax{+XXX XXX XXX}
\email{jsnow@westeros.org}
\extrainfo{\href{www.mywebsite.com}{www.mywebsite.com}}
% You now need to create a folder "image" and upload the pictures you need for your document.
\graphicspath{ {image/} } % define the path of where the images are stored
\photo[64pt]{picture1} % add your picture
\quote{You can add a short summary of your career and future plans here}

% Other packages: 
% package to split the page in columns
\usepackage{multicol} 

% package to compact lists
\usepackage{enumitem}

% Package and code to add the page numbering in the footer.
\usepackage{lastpage}
\rfoot{\addressfont\itshape\textcolor{gray}{Page \thepage\ of \pageref{LastPage}}}
% Note that in a standard document type such as an article, you would only need to add \pagenumbering{arabic} to have the page numbering.

% package to manage pictures
\usepackage {graphicx}


% ------- Your CV starts here --------

\begin{document} 
\maketitle

\section{EDUCATION}
	\cventry{year-year}{Master in Ecotoxicology}{University of Koblenz-Landau}{Germany}
\closesection{}

\section{KEY SKILLS}
	\subsection{Programming}
		\begin{multicols}{3}
		\begin{itemize}
			\item{\LaTeX}
			\item{R}
			\item{Markdown}
			\item{Python}
			\item{Java}
		\end{itemize}
		\end{multicols}
		
\closesection{}

\section{PUBLICATIONS}% not printing the publications somehow
	\nocite{*}
	\printbibliography[heading=none]
\closesection

\section{RESEARCH}
Type the text that you want to have in your research statement here.
\par % this command line introduces a new paragraph. Try to remove it and see what happens.
You can \textit{see} \underline{that} \textbf{we can modify} the \textbf{\underline{\textit{appearance}}} of the text as we want.
\closesection

\section{HONOURS AND AWARDS}
\section{FUNDING}
Similarly to the none number list using \textit{ITEMIZE}, we can also create numbered lists:

    \begin{enumerate}
        \item Funding 1
        \item Funding 2
    \end{enumerate}

\closesection


\section{ADD ANY OTHER SECTION}
% I now show you how to insert an image in a section or paragraph
% define where we want the image place
% h: Here, t: Top of the page, b: Bottom, p: Separate page, !: improves the float
We now insert a picture: % we can see here that the double slash insert a line spacing.

\centering
\includegraphics[width=8cm]{picture1.png} % there is no need to adjust the height as this is done automatically.
\end{document}